% Created 2021-12-09 Thu 16:14
% Intended LaTeX compiler: pdflatex
\documentclass[11pt]{article}
\usepackage[utf8]{inputenc}
\usepackage[T1]{fontenc}
\usepackage{graphicx}
\usepackage{longtable}
\usepackage{wrapfig}
\usepackage{rotating}
\usepackage[normalem]{ulem}
\usepackage{amsmath}
\usepackage{amssymb}
\usepackage{capt-of}
\usepackage{hyperref}
\author{Ivan Dyachenko}
\date{\today}
\title{\href{https://codeforces.com/edu/course/2/lesson/7}{Disjoint Sets Union}}
\hypersetup{
 pdfauthor={Ivan Dyachenko},
 pdftitle={\href{https://codeforces.com/edu/course/2/lesson/7}{Disjoint Sets Union}},
 pdfkeywords={},
 pdfsubject={},
 pdfcreator={Emacs 29.0.50 (Org mode 9.5.1)}, 
 pdflang={English}}
\begin{document}

\maketitle
\tableofcontents


\section{Part 1}
\label{sec:org79b3ab2}
Disjoint sets union (DSU) is a data structure that supports disjoint sets on \(n\) elements and
allows two type of queries:
\begin{itemize}
\item \texttt{get(a)} - return the identifier of the set to which \(a\) belongs to;
\item \texttt{union(a, b)} - join two sets that contain \(a\) and \(b\).
\end{itemize}

For example, when we call \texttt{get(a)} and \texttt{get(b)}, we can compare whether \(a\) and \(b\) are in the
same set.

What is the simplest way to define the identifier of the set? - As an identifier, we can choose
the leader of the set.

Let us maintain an array \(p\), where \(p[a]\) is the identifier (the leader) of a set to which \(a\)
belongs to.

Let us consider the pseudo-code of two functions:
\begin{verbatim}
init():
  p = new int[n]
  for i in 1..n:
    p[i] = i

get(a):
  return p[a]

union(a, b):
  a = p[a]
  b = p[b]
  for i in 1..n:
    if p[i] == a:
      p[i] = b
\end{verbatim}

The function \texttt{get(a)} just returns the leader of a set, and the function \texttt{union(a, b)} takes the
leaders of both sets and set \(b\) as a leader of elements with the leader \(a\).

Unfortunately, this algorithm is too slow: \texttt{get} works in \(\mathcal{O}(1)\), however, \texttt{union} works
in \(\mathcal{O}(n)\). Is there a way to improve the algorithm?

Let us consider the simplest idea - let us iterate not over all the elements, but over the
elements with the leader \(a\). For that, for each leader, we will maintain a linked list
\(l[a]\). When we have to unite two sets, we will just link two lists together.

\begin{verbatim}
init():
  p = new int[n]
  l = new List[n]
  for i in 1..n:
    p[i] = i
    l[i] = { i }

get(a):
  return p[a]

union(a, b):
  a = p[a]
  b = p[b]
  for x in l[a]:
    p[x] = b
  l[b].append(l[a])
\end{verbatim}

Now, \texttt{get(a)} works in \(\mathcal{O}(1)\) and \texttt{union(a, b)} works in
\(\matcal{O}(|l[a]|)\). Unfortunately, this complexity is not good enough: it is possible to find an
execution such that \texttt{union} will work in \(\matchal{O}(n)\) in amortization. Consider this
execution:
\begin{itemize}
\item \texttt{union(1, 2)}, where \(|l[1]| = 1\) and \(|l[2]| = 1\),
\item \texttt{union(2, 3)}, where \(|l[2]| = 2\) and \(|l[3]| = 1\),
\item \texttt{union(3, 4)}, where \(|l[3]| = 3\) and \(|l[4]| = 1\),
\item and so on.
\end{itemize}

All operations in total work in \(1 + 2 + 3 + ... + (n - 1) = \mathcal{O}(n^2)\), and, thus, \texttt{union}
still works in \(\mathcal{O}(n)\). How to improve it? Note that the main problem is that we always
join the first set to the second one. But what if we join the smallest set to the largest? Then,
the code of \texttt{union} becomes the following:
\begin{verbatim}
union(a, b):
  a = p[a]
  b = p[b]
  if size(l[a]) > size(l[b]):
    swap(a, b)
  for x in l[a]:
    p[x] = b
  l[b].append(l[a])
\end{verbatim}

We compare two sets and if the set \(a\) is larger than the set \(b\) we swap them. Note that we can
implement \texttt{size(l[a])} in \(\mathcal{O}(1)\) - for that we have to store the size of the list
separately. How fast does this algorithm work? \texttt{get(a)} still works in \(\mathcal{O}(1)\), but did
we improve \texttt{union(a, b)}? Let us calculate how many times we changed the leader of \(x\), i.e., the
algorithm performs line \texttt{p[x] = b}. The first time we changed the leader of \(x\) is when we unite
it with the larger set. This means that the size of the union is at least \(2\). The second time we
changed the leader of \(x\) is when we unite the set with the larger set of size at leaset \(2\). This
means that the size of the union is at least \(4\). And so on. We change the leader of \(x\) only when
we unite with the larger set. Since we unite all the sets together, we perform \(\mathcal{O}(\log
  n)\) changes per element and, thus, the total cost is \(\mathcal{O}(n \log n)\). Since, there are
\(n - 1\) \texttt{union} operations, each of them works in \(\mathcal{O}(\log n)\) in amortization.

\section{Part 2}
\label{sec:orga4cae89}
In the previous part we explained how to implement \texttt{get(a)} in \(\mathcal{O}(1)\) and \texttt{union(a, b)}
in \(\mathcal{O}(\log n)\) amortized. But can we reduce the complexity of \texttt{union}, while slowdown
\texttt{get} a little bit? It appears to be possible, but we should treat the data structure another
way. We need to store the elements another way rather than in linked lists - for example, we can
store them in trees. We are already given an array \(p\): let us store there a parent of an element
in a tree. If \(p[a]\) is equal to \(a\), then \(a\) is a root and a leader of the corresponding
set. Initially, each element is a root of its own set, i.e., \(p[a] = a\). To implement \texttt{get}, we
just simply need to follow the parent links until we find the root. To implement \texttt{union}, we need
to find the leaders of both sets and link one set to another.
\begin{verbatim}
get(a):
  while a != p[a]:
    a = p[a]
  return a

union(a, b):
  a = get(a)
  b = get(b)
  p[a] = b
\end{verbatim}

Unfortunately, such an algorithm is subject to the problem discussed in the previous part: the
total time of \texttt{union} operations can reach \(\mathcal{O}(n^2)\). But we already know how to solve
such an issue. For that, we have to join the smaller set to the larger one. When we unite two
sets, elements of the smaller set now have one more link to the root. It is not hard to show that
for each element the total number of links to pass to reach the root cannot exceed
\(\mathcal{O}(\log n)\). Thus, we get that \texttt{get} and \texttt{union} works in \(\mathcal{O}(\log n)\) (not in
amortization). It is pretty simple to implement.
\begin{verbatim}
union(a, b):
  a = get(a)
  b = get(b)
  if size[a] > size[b]:
    swap(a, b)
  p[a] = b
  size[b] += size[a]
\end{verbatim}

How to improve the algorithm further? Note, that when we call \texttt{get} we find the root. Then, it is
reasonable to update \(p[a]\) to point to the root, so that next \texttt{get} will work faster. Operation
\texttt{get} becomes the following.
\begin{verbatim}
get(a):
  if p[a] != a:
    p[a] = get(p[a])
  return p[a]
\end{verbatim}

We rewrote the function in a recursive manner. If \(a\) is a root, then the result is \(p[a]\),
otherwise, we set \(p[a]\) to the root. This heuristic is named \textbf{path-compression}.

It appears that if we apply both heuristics: the path-compression heuristic and the
link-small-to-large heuristic, we get that \texttt{get} and \texttt{union} work in \(\mathcal{O}(\alpha(m, n))\)
time amortized, where \(\alpha(m, n)\) is the inverse Ackermann function, \(m\) is the number of
performed operations \texttt{get} and \(n\) is the number of elements.

To give the intuition on how slow the inverse Ackermann function rises, we look at
\(\mathcal{O}(\log^* n)\) that rises a little bit faster. This function means how many times we
should take the binary logarithm of \(n\) to get a value smaller than one. Consider an
example. Suppose we take a very large number \(2^{65536}\) and calculate its \(\log^*\). \(2^{65536}
  \rightarrow 65536=2^{16} \rightarrow 16=2^4 \rightarrow 4 = 2^2 \rightarrow 2 \rightarrow 1
  \rightarrow 0\). In total we get that \(\log^* 2^{65536} = 6\). So, we can suppose that for all
reasonable \(n\) this function is almost constant, while the inverse Ackermann function rises even
slower.
\end{document}